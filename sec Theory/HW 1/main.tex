\documentclass{article}

\usepackage{amsmath}

\begin{document}

\title{Homework 1}
\author{Pupipat Singkhorn}
\date{\today}
\maketitle

Read about `continuation', what can it do? It uses `functional value' to capture the history of the control flow (or the future of computation). You can `dig-deep' or `general know' about this subject. Write a paragraph is enough. Use any tools you want.

\section*{What is Continuation?}
\quad Continuation is a programming concept that represents the future of computation or the history of control flow in a program. It captures the program's control state at a specific point, allowing the program to be paused and later resumed from that point. Continuations are often used in functional programming languages to implement advanced control flow mechanisms.

\section*{What Can It Do?}
\quad Continuations use functional values to capture the history of the control flow or the future of computation. They can be used to implement features such as non-local exits, backtracking, coroutines, generators, and asynchronous programming. Continuations essentially encapsulate the state of a computation, enabling the program to capture and manipulate its own control flow.

\begin{thebibliography}{9}
\bibitem{wiki-continuation}
Wikipedia contributors. (2022, March 13). Continuation. In \textit{Wikipedia, The Free Encyclopedia}. Retrieved 07:46, April 8, 2024, from \texttt{https://en.wikipedia.org/wiki/Continuation}
\bibitem{}
This explanation was informed by a conversation with a large language model (e.g., ChatGPT or Gemini) on \today.
\end{thebibliography}


\end{document}